\documentclass[11pt]{article}
\usepackage{amsmath, amssymb, amsthm}
\usepackage{geometry}
\usepackage{algorithm}
\usepackage{algorithmic}
\usepackage{graphicx}
\usepackage{hyperref}

\geometry{margin=1in}

\title{Penalty Method for Enforcing Zero Velocity in a Subdomain:\\
Application to the Driven Cavity Problem}
\author{}
\date{}

\begin{document}

\maketitle

\section{Strong Formulation of the Incompressible Navier-Stokes Equations}

The incompressible Navier-Stokes equations in a domain $\Omega \subset \mathbb{R}^2$ are given by:

\begin{align}
    \frac{\partial \mathbf{u}}{\partial t} + (\mathbf{u} \cdot \nabla) \mathbf{u} &= -\nabla p + \nu \Delta \mathbf{u} + \mathbf{f} \quad \text{in } \Omega \times (0, T] \label{eq:momentum}\\
    \nabla \cdot \mathbf{u} &= 0 \quad \text{in } \Omega \times (0, T] \label{eq:continuity}
\end{align}

where:
\begin{itemize}
    \item $\mathbf{u} = (u, v)^T$ is the velocity field
    \item $p$ is the pressure (divided by density)
    \item $\nu$ is the kinematic viscosity
    \item $\mathbf{f}$ is the body force
\end{itemize}

In component form, for $\mathbf{u} = (u, v)$:
\begin{align}
    \frac{\partial u}{\partial t} + u\frac{\partial u}{\partial x} + v\frac{\partial u}{\partial y} &= -\frac{\partial p}{\partial x} + \nu \left(\frac{\partial^2 u}{\partial x^2} + \frac{\partial^2 u}{\partial y^2}\right) \\
    \frac{\partial v}{\partial t} + u\frac{\partial v}{\partial x} + v\frac{\partial v}{\partial y} &= -\frac{\partial p}{\partial y} + \nu \left(\frac{\partial^2 v}{\partial x^2} + \frac{\partial^2 v}{\partial y^2}\right) \\
    \frac{\partial u}{\partial x} + \frac{\partial v}{\partial y} &= 0
\end{align}

\section{The Penalty Method for Immersed Boundaries}

\subsection{Problem Statement}

We want to enforce $\mathbf{u} = \mathbf{0}$ in a subdomain $\Omega_s \subset \Omega$ (e.g., a circular region representing a solid obstacle). This is commonly used in:
\begin{itemize}
    \item Immersed boundary methods
    \item Fluid-structure interaction
    \item Modeling obstacles without body-fitted meshes
\end{itemize}

\subsection{Penalty Formulation}

The penalty method adds a forcing term to the momentum equation that penalizes any deviation from the desired velocity in the solid region:

\begin{equation}
    \frac{\partial \mathbf{u}}{\partial t} + (\mathbf{u} \cdot \nabla) \mathbf{u} = -\nabla p + \nu \Delta \mathbf{u} - \eta \chi_s(\mathbf{x}) \mathbf{u}
    \label{eq:penalized_momentum}
\end{equation}

where:
\begin{itemize}
    \item $\eta > 0$ is the \textbf{penalty coefficient} (large value)
    \item $\chi_s(\mathbf{x})$ is the \textbf{characteristic function} of the solid region:
    \begin{equation}
        \chi_s(\mathbf{x}) = 
        \begin{cases}
            1 & \text{if } \mathbf{x} \in \Omega_s \\
            0 & \text{if } \mathbf{x} \notin \Omega_s
        \end{cases}
    \end{equation}
\end{itemize}

\subsection{Physical Interpretation}

The penalty term $-\eta \chi_s \mathbf{u}$ acts as a \textbf{Darcy-like drag force} that resists fluid motion in the solid region. As $\eta \to \infty$, the velocity $\mathbf{u} \to \mathbf{0}$ in $\Omega_s$.

The modified momentum equation can be rewritten as:
\begin{equation}
    \frac{\partial \mathbf{u}}{\partial t} = \underbrace{-(\mathbf{u} \cdot \nabla) \mathbf{u}}_{\text{Convection}} \underbrace{- \nabla p}_{\text{Pressure}} + \underbrace{\nu \Delta \mathbf{u}}_{\text{Diffusion}} \underbrace{- \eta \chi_s \mathbf{u}}_{\text{Penalty}}
\end{equation}

\subsection{Definition of the Solid Region}

For a circular obstacle centered at $(x_c, y_c)$ with radius $R$:
\begin{equation}
    \chi_s(\mathbf{x}) = 
    \begin{cases}
        1 & \text{if } (x - x_c)^2 + (y - y_c)^2 \leq R^2 \\
        0 & \text{otherwise}
    \end{cases}
\end{equation}

\section{Numerical Discretization}

\subsection{Projection Method (Chorin's Splitting)}

The original code uses a projection method. The time advancement consists of two steps:

\textbf{Step 1: Predictor (compute intermediate velocity $\tilde{\mathbf{u}}$)}
\begin{align}
    \frac{\tilde{u} - u^n}{\Delta t} &= -\left(u\frac{\partial u}{\partial x} + v\frac{\partial u}{\partial y}\right)^n + \nu \Delta u^n \\
    \frac{\tilde{v} - v^n}{\Delta t} &= -\left(u\frac{\partial v}{\partial x} + v\frac{\partial v}{\partial y}\right)^n + \nu \Delta v^n
\end{align}

\textbf{Step 2: Pressure Projection (enforce incompressibility)}
\begin{align}
    \Delta p^{n+1} &= \frac{1}{\Delta t} \nabla \cdot \tilde{\mathbf{u}} \\
    \mathbf{u}^{n+1} &= \tilde{\mathbf{u}} - \Delta t \nabla p^{n+1}
\end{align}

\subsection{Adding the Penalty Term}

\subsubsection{Explicit Treatment (Unstable)}

A naive explicit discretization would be:
\begin{equation}
    \tilde{u} = u^n + \Delta t \left( \text{Conv} + \text{Diff} - \eta \chi_s u^n \right)
\end{equation}

However, for large $\eta$, the term $\Delta t \cdot \eta$ becomes very large (e.g., $\Delta t = 0.01$, $\eta = 10^4$ gives $\Delta t \cdot \eta = 100$), causing \textbf{numerical instability}.

\subsubsection{Implicit Treatment (Stable)}

To ensure stability, we treat the penalty term implicitly:
\begin{equation}
    \frac{\tilde{u} - u^n}{\Delta t} = \text{Conv}^n + \text{Diff}^n - \eta \chi_s \tilde{u}
\end{equation}

Rearranging:
\begin{equation}
    \tilde{u} + \Delta t \cdot \eta \chi_s \tilde{u} = u^n + \Delta t \left( \text{Conv}^n + \text{Diff}^n \right)
\end{equation}

\begin{equation}
    \tilde{u} \left(1 + \Delta t \cdot \eta \chi_s \right) = u^n + \Delta t \left( \text{Conv}^n + \text{Diff}^n \right)
\end{equation}

Therefore:
\begin{equation}
    \boxed{\tilde{u} = \frac{u^n + \Delta t \left( \text{Conv}^n + \text{Diff}^n \right)}{1 + \Delta t \cdot \eta \chi_s}}
    \label{eq:implicit_penalty}
\end{equation}

\textbf{Key observations:}
\begin{itemize}
    \item In the fluid region ($\chi_s = 0$): $\tilde{u} = u^n + \Delta t (\text{Conv} + \text{Diff})$ (unchanged)
    \item In the solid region ($\chi_s = 1$): $\tilde{u} = \frac{u^n + \Delta t (\text{Conv} + \text{Diff})}{1 + \Delta t \cdot \eta}$
    \item As $\eta \to \infty$: $\tilde{u} \to 0$ in the solid region
    \item This is \textbf{unconditionally stable} for any $\eta > 0$
\end{itemize}

\section{Algorithm Summary}

\begin{algorithm}[H]
\caption{Projection Method with Penalty}
\begin{algorithmic}[1]
\STATE \textbf{Input:} $u^n, v^n$, penalty mask $\chi_s$, penalty coefficient $\eta$
\STATE Apply boundary conditions
\STATE Compute convection and diffusion terms
\STATE Compute intermediate velocity (without pressure):
\[
\tilde{u}^* = u^n + \Delta t (\text{Conv}_u + \text{Diff}_u)
\]
\[
\tilde{v}^* = v^n + \Delta t (\text{Conv}_v + \text{Diff}_v)
\]
\STATE \textbf{Apply implicit penalty:}
\[
\tilde{u} = \frac{\tilde{u}^*}{1 + \Delta t \cdot \eta \cdot \chi_s}
\]
\[
\tilde{v} = \frac{\tilde{v}^*}{1 + \Delta t \cdot \eta \cdot \chi_s}
\]
\STATE Solve pressure Poisson equation: $\Delta p = \frac{1}{\Delta t} \nabla \cdot \tilde{\mathbf{u}}$
\STATE Project to divergence-free velocity:
\[
u^{n+1} = \tilde{u} - \Delta t \frac{\partial p}{\partial x}, \quad v^{n+1} = \tilde{v} - \Delta t \frac{\partial p}{\partial y}
\]
\end{algorithmic}
\end{algorithm}

\section{Code Modifications}

\subsection{New Function: \texttt{define\_penalty\_region}}

This function creates the characteristic function $\chi_s$ on the computational grid:

\begin{verbatim}
def define_penalty_region(nx, ny, region_type='circle', 
                          x_center=0.5, y_center=0.5, radius=0.1):
    mask = np.zeros((nx, ny))
    x = np.linspace(0, 1, nx)
    y = np.linspace(0, 1, ny)
    xx, yy = np.meshgrid(x, y)
    
    if region_type == 'circle':
        mask = ((xx - x_center)**2 + (yy - y_center)**2 <= radius**2)
    
    return mask.astype(float)
\end{verbatim}

\subsection{Modified \texttt{solve} Function}

Added penalty method setup:
\begin{verbatim}
# Penalty method setup
penalty_mask_u = np.zeros((ny, nx-1))
penalty_mask_v = np.zeros((ny-1, nx))
penalty_coeff = 1e4  # eta

if penalty_method and penalty_param is not None:
    region_type, region_args = penalty_param
    base_mask = define_penalty_region(nx, ny, region_type, *region_args)
    # Masks adapted to staggered grid
    penalty_mask_u = base_mask[:, :-1]  # u-component grid
    penalty_mask_v = base_mask[:-1, :]  # v-component grid
\end{verbatim}

Modified momentum equations with implicit penalty:
\begin{verbatim}
# Compute intermediate velocity
ut[1:ny+1, 2:nx+1] = u[1:ny+1, 2:nx+1] + dt * (convection + diffusion)

# Apply implicit penalty (Equation 12)
if penalty_method:
    ut[1:ny+1, 2:nx+1] = ut[1:ny+1, 2:nx+1] / (1.0 + dt * penalty_coeff * penalty_mask_u)
\end{verbatim}

\subsection{Staggered Grid Considerations}

On a staggered (MAC) grid:
\begin{itemize}
    \item Pressure is defined at cell centers: $p_{i,j}$
    \item $u$-velocity is defined at vertical cell faces: $u_{i+1/2,j}$
    \item $v$-velocity is defined at horizontal cell faces: $v_{i,j+1/2}$
\end{itemize}

The penalty mask must be interpolated to the appropriate velocity locations:
\begin{align}
    \chi_s^u &= \chi_s[:, :-1] \quad \text{(shape: } n_y \times (n_x-1) \text{)} \\
    \chi_s^v &= \chi_s[:-1, :] \quad \text{(shape: } (n_y-1) \times n_x \text{)}
\end{align}

\section{Parameter Selection}

\subsection{Penalty Coefficient $\eta$}

\begin{itemize}
    \item \textbf{Too small} ($\eta \sim 1$): Velocity not effectively suppressed
    \item \textbf{Optimal range} ($\eta \sim 10^3 - 10^5$): Good enforcement with stability
    \item \textbf{Very large} ($\eta \to \infty$): Exact enforcement, but may cause stiff behavior
\end{itemize}

With implicit treatment, larger values of $\eta$ can be used without stability issues.

\subsection{Error Estimate}

The velocity in the penalized region satisfies approximately:
\begin{equation}
    |\mathbf{u}|_{\Omega_s} \sim \mathcal{O}\left(\frac{1}{\eta \Delta t}\right)
\end{equation}

For $\eta = 10^4$ and $\Delta t = 0.01$: $|\mathbf{u}|_{\Omega_s} \sim \mathcal{O}(10^{-2})$

\section{Extensions}

\subsection{Prescribing Non-Zero Velocity}

To enforce $\mathbf{u} = \mathbf{u}_d$ instead of $\mathbf{u} = \mathbf{0}$:
\begin{equation}
    \frac{\partial \mathbf{u}}{\partial t} + (\mathbf{u} \cdot \nabla) \mathbf{u} = -\nabla p + \nu \Delta \mathbf{u} - \eta \chi_s (\mathbf{u} - \mathbf{u}_d)
\end{equation}

The implicit discretization becomes:
\begin{equation}
    \tilde{u} = \frac{u^n + \Delta t (\text{Conv} + \text{Diff}) + \Delta t \cdot \eta \chi_s u_d}{1 + \Delta t \cdot \eta \chi_s}
\end{equation}

\subsection{CFL Stability Constraint}

\textbf{Critical Issue:} When prescribing large velocities $|\mathbf{u}_d| \gg 1$, the explicit treatment of the convection term imposes a stability constraint:

\begin{equation}
    \text{CFL} = \frac{|\mathbf{u}|_{\max} \Delta t}{\min(\Delta x, \Delta y)} \leq C_{\text{crit}}
\end{equation}

where $C_{\text{crit}} \approx 0.5$ for explicit schemes. This means:

\begin{equation}
    \Delta t \leq \frac{C_{\text{crit}} \cdot \min(\Delta x, \Delta y)}{|\mathbf{u}_d|}
\end{equation}

\textbf{Example:} For $u_d = -2.0$, $\Delta x = 0.02$, we need:
\begin{equation}
    \Delta t \leq \frac{0.5 \times 0.02}{2.0} = 0.005
\end{equation}

If $\Delta t = 0.01$ is used, the solution becomes unstable and diverges.

\subsubsection{Solutions for High Velocity Prescription}

\textbf{Option 1: Reduce Time Step (Simplest)}

Adaptively adjust the time step based on the maximum velocity:
\begin{equation}
    \Delta t = C_{\text{CFL}} \frac{\min(\Delta x, \Delta y)}{\max(|\mathbf{u}_d|, |\mathbf{u}|_{\text{flow}})}
\end{equation}

where $C_{\text{CFL}} = 0.3 - 0.5$ for safety.

\textbf{Option 2: Semi-Implicit Convection}

Treat the penalty-induced velocity implicitly in the convection term:
\begin{equation}
    \text{Conv} = -\left(u\frac{\partial u}{\partial x} + v\frac{\partial u}{\partial y}\right)
\end{equation}

In the penalty region, use $u_d$ for the convection velocity instead of $u^n$.

\textbf{Option 3: Ramp Up Velocity Gradually}

Instead of immediately imposing $u_d = -2$, use a smooth ramp:
\begin{equation}
    u_d(t) = u_d^{\text{final}} \cdot \min\left(1, \frac{t}{T_{\text{ramp}}}\right)
\end{equation}

This allows the flow to adjust gradually, avoiding initial transients.

\subsection{Smooth Transition}

Instead of a sharp mask, use a smooth transition:
\begin{equation}
    \chi_s(\mathbf{x}) = \frac{1}{2}\left(1 - \tanh\left(\frac{d(\mathbf{x}) - R}{\epsilon}\right)\right)
\end{equation}
where $d(\mathbf{x})$ is the distance from the obstacle center and $\epsilon$ controls the transition width.

\end{document}
