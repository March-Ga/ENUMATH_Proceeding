\documentclass{article}
\usepackage{amsmath,amssymb}

\begin{document}

\section{Residuo del solido con inerzia nel metodo immerso}

\subsection{Formulazione del problema}

Consideriamo un problema FSI con approccio \emph{immersed domain}: il fluido è definito su tutto il dominio $\Omega$, mentre il solido occupa un sottodominio $\Omega_s \subset \Omega$. L'accoppiamento è gestito tramite moltiplicatori di Lagrange distribuiti su $\Omega_s$.

\subsubsection{Equazioni governanti}

\textbf{Fluido} (Navier-Stokes incomprimibile su $\Omega$):
\begin{align}
\rho_f \frac{\partial \mathbf{u}_f}{\partial t} + \rho_f (\mathbf{u}_f \cdot \nabla)\mathbf{u}_f - \nabla \cdot \boldsymbol{\sigma}_f &= \mathbf{f}_{\text{ext}} \\
\nabla \cdot \mathbf{u}_f &= 0
\end{align}

\textbf{Solido} (Elastodinamica su $\Omega_s$):
\begin{equation}
\rho_s \frac{\partial \mathbf{v}_s}{\partial t} = \nabla \cdot \mathbf{P}(\mathbf{d}_s) + \mathbf{b}_s
\end{equation}
dove $\mathbf{d}_s$ è lo spostamento, $\mathbf{v}_s = \partial \mathbf{d}_s / \partial t$ la velocità, e $\mathbf{P}$ il primo tensore di Piola-Kirchhoff.

\subsection{Schema partizionato}

Ad ogni timestep, l'algoritmo iterativo è:
\begin{enumerate}
    \item Imporre al solido la velocità del fluido: $\mathbf{v}_s = T(\mathbf{u}_f)$
    \item Calcolare il residuo del solido $\mathbf{R}_s$ dovuto alla velocità imposta
    \item Trasferire il residuo al fluido: $\mathbf{f}_{\text{ext}} = T^T(\mathbf{R}_s)$
    \item Risolvere le equazioni del fluido con $\mathbf{f}_{\text{ext}}$
    \item Iterare fino a convergenza
\end{enumerate}

\subsection{Calcolo del residuo solido con inerzia}

\subsubsection{Dalla velocità imposta all'accelerazione}

Quando imponiamo la velocità $\mathbf{v}_s^{n+1} = T(\mathbf{u}_f^{n+1})$, dobbiamo ricostruire l'accelerazione per il termine inerziale. Usando uno schema di Newmark o BDF:

\textbf{Schema BDF1 (Backward Euler):}
\begin{equation}
\mathbf{a}_s^{n+1} = \frac{\mathbf{v}_s^{n+1} - \mathbf{v}_s^{n}}{\Delta t}
\end{equation}

\textbf{Schema BDF2:}
\begin{equation}
\mathbf{a}_s^{n+1} = \frac{3\mathbf{v}_s^{n+1} - 4\mathbf{v}_s^{n} + \mathbf{v}_s^{n-1}}{2\Delta t}
\end{equation}

Lo spostamento si aggiorna analogamente:
\begin{equation}
\mathbf{d}_s^{n+1} = \mathbf{d}_s^{n} + \Delta t \, \mathbf{v}_s^{n+1} \quad \text{(BDF1)}
\end{equation}

\subsubsection{Definizione del residuo}

Il residuo dell'elastodinamica è:
\begin{equation}
\boxed{\mathbf{R}_s = \rho_{\text{eff}} \, \mathbf{a}_s^{n+1} - \nabla \cdot \mathbf{P}(\mathbf{d}_s^{n+1})}
\end{equation}

dove $\rho_{\text{eff}}$ è la densità effettiva da determinare.

\subsection{Perché usare $\rho_s - \rho_f$ invece di $\rho_s$?}

Questa è la domanda cruciale. La risposta dipende dalla \textbf{formulazione del fluido nella regione solida}.

\subsubsection{Il problema della doppia inerzia}

Nel metodo immerso, il fluido è risolto su \textbf{tutto} il dominio $\Omega$, inclusa la regione $\Omega_s$. Quindi l'equazione del fluido in $\Omega_s$ contiene già un termine inerziale:
\begin{equation}
\rho_f \frac{\partial \mathbf{u}_f}{\partial t} \quad \text{in } \Omega_s
\end{equation}

Se il residuo del solido usasse la densità piena $\rho_s$:
\begin{equation}
\mathbf{R}_s = \rho_s \, \mathbf{a}_s - \nabla \cdot \mathbf{P}
\end{equation}
e questo venisse trasferito al fluido come forza esterna, avremmo:
\begin{equation}
\underbrace{\rho_f \, \mathbf{a}_f}_{\text{inerzia fluido in } \Omega_s} + \underbrace{\rho_s \, \mathbf{a}_s}_{\text{da } T^T(\mathbf{R}_s)} = \text{altre forze}
\end{equation}

Poiché imponiamo $\mathbf{v}_s = \mathbf{u}_f$ (e quindi $\mathbf{a}_s = \mathbf{a}_f$) in $\Omega_s$, l'inerzia totale sarebbe:
\begin{equation}
(\rho_f + \rho_s) \, \mathbf{a} \quad \textbf{SBAGLIATO!}
\end{equation}

\subsubsection{La correzione}

Per ottenere la corretta inerzia $\rho_s \, \mathbf{a}$ nella regione solida, dobbiamo \textbf{sottrarre} l'inerzia del fluido già presente. Quindi:

\begin{equation}
\boxed{\mathbf{R}_s = (\rho_s - \rho_f) \, \mathbf{a}_s - \nabla \cdot \mathbf{P}(\mathbf{d}_s)}
\end{equation}

In questo modo:
\begin{equation}
\underbrace{\rho_f \, \mathbf{a}}_{\text{dal fluido}} + \underbrace{(\rho_s - \rho_f) \, \mathbf{a}}_{\text{dal residuo}} = \rho_s \, \mathbf{a} \quad \checkmark
\end{equation}

\subsection{Formulazione completa}

\subsubsection{Residuo del solido}

\begin{equation}
\boxed{\mathbf{R}_s^{n+1} = (\rho_s - \rho_f) \, \frac{\mathbf{v}_s^{n+1} - \mathbf{v}_s^{n}}{\Delta t} - \nabla \cdot \mathbf{P}(\mathbf{d}_s^{n+1})}
\end{equation}

\subsubsection{Algoritmo partizionato completo}

Data la soluzione al tempo $t^n$, per trovare la soluzione a $t^{n+1}$:

\begin{enumerate}
    \item \textbf{Inizializzazione}: $\mathbf{u}_f^{(0)} = \mathbf{u}_f^{n}$, $k=0$
    
    \item \textbf{Loop iterativo}: per $k = 0, 1, 2, \ldots$ fino a convergenza:
    \begin{enumerate}
        \item Trasferire velocità al solido:
        \begin{equation}
        \mathbf{v}_s^{(k+1)} = T(\mathbf{u}_f^{(k)})
        \end{equation}
        
        \item Aggiornare spostamento solido:
        \begin{equation}
        \mathbf{d}_s^{(k+1)} = \mathbf{d}_s^{n} + \Delta t \, \mathbf{v}_s^{(k+1)}
        \end{equation}
        
        \item Calcolare accelerazione:
        \begin{equation}
        \mathbf{a}_s^{(k+1)} = \frac{\mathbf{v}_s^{(k+1)} - \mathbf{v}_s^{n}}{\Delta t}
        \end{equation}
        
        \item Calcolare residuo solido (con densità corretta):
        \begin{equation}
        \mathbf{R}_s^{(k+1)} = (\rho_s - \rho_f) \, \mathbf{a}_s^{(k+1)} - \nabla \cdot \mathbf{P}(\mathbf{d}_s^{(k+1)})
        \end{equation}
        
        \item Trasferire al fluido:
        \begin{equation}
        \mathbf{f}_{\text{ext}}^{(k+1)} = T^T(\mathbf{R}_s^{(k+1)})
        \end{equation}
        
        \item Risolvere Navier-Stokes:
        \begin{equation}
        \rho_f \frac{\mathbf{u}_f^{(k+1)} - \mathbf{u}_f^{n}}{\Delta t} + \rho_f (\mathbf{u}_f^{(k+1)} \cdot \nabla)\mathbf{u}_f^{(k+1)} = \nabla \cdot \boldsymbol{\sigma}_f + \mathbf{f}_{\text{ext}}^{(k+1)}
        \end{equation}
        
        \item Verificare convergenza: $\|\mathbf{u}_f^{(k+1)} - \mathbf{u}_f^{(k)}\| < \text{tol}$
    \end{enumerate}
    
    \item \textbf{Aggiornamento}: $\mathbf{u}_f^{n+1} = \mathbf{u}_f^{(k+1)}$, $\mathbf{d}_s^{n+1} = \mathbf{d}_s^{(k+1)}$, ecc.
\end{enumerate}

\subsection{Osservazioni importanti}

\begin{itemize}
    \item Se $\rho_s = \rho_f$ (solido con stessa densità del fluido), il termine inerziale nel residuo si annulla e rimane solo il contributo elastico: si recupera il caso quasi-statico.
    
    \item Se $\rho_s > \rho_f$ (solido più denso), il termine $(\rho_s - \rho_f)\mathbf{a}_s$ è positivo e aggiunge inerzia.
    
    \item Se $\rho_s < \rho_f$ (solido meno denso, es. bolla), il termine è negativo, riducendo l'inerzia effettiva—fisicamente corretto!
    
    \item Questo approccio è noto come \textbf{added mass correction} o \textbf{density difference formulation} nella letteratura sui metodi immersi.
\end{itemize}

\subsection{Trasferimento del residuo al fluido e scalatura}

\subsubsection{Unità di misura del residuo solido}

Il residuo calcolato sul solido è:
\begin{equation}
\mathbf{R}_s = (\rho_s - \rho_f) \, \mathbf{a}_s - \nabla \cdot \mathbf{P}(\mathbf{d}_s)
\end{equation}

Dopo discretizzazione FEM, il sistema algebrico è:
\begin{equation}
\mathbf{R}_s^h = M_s \, \mathbf{a}_s - \mathbf{F}_{\text{int}}(\mathbf{d}_s)
\end{equation}
dove:
\begin{itemize}
    \item $M_s$ è la matrice di massa [kg]
    \item $\mathbf{F}_{\text{int}}(\mathbf{d}_s)$ è il \textbf{vettore} delle forze interne elastiche [N]
\end{itemize}

Nel caso di elasticità lineare, il vettore delle forze interne si scrive come:
\begin{equation}
\mathbf{F}_{\text{int}}(\mathbf{d}_s) = K_s \, \mathbf{d}_s
\end{equation}
dove $K_s$ è la \textbf{matrice} di rigidezza [N/m]. Nel caso non lineare (grandi deformazioni), $\mathbf{F}_{\text{int}}$ è una funzione non lineare di $\mathbf{d}_s$.

\textbf{Le unità di misura sono:}
\begin{itemize}
    \item $\mathbf{R}_s^h$ ha unità di \textbf{forza} [N] (forza nodale concentrata)
    \item Questo perché la formulazione debole integra sul volume: $\int_{\Omega_s} \phi_i \, (\rho_s \mathbf{a} - \nabla \cdot \mathbf{P}) \, dV$
\end{itemize}

Nel caso quasi-statico (senza inerzia), il residuo è semplicemente:
\begin{equation}
\boldsymbol{\lambda} = K_s \, \mathbf{d}_s = \mathbf{F}_{\text{int}}
\end{equation}
e $\boldsymbol{\lambda}$ ha unità di \textbf{forza nodale} [N].

\subsubsection{L'operatore di trasferimento $B$ e la sua trasposta}

Definiamo l'operatore $B$ che interpola la velocità del fluido sui nodi del solido:
\begin{equation}
\mathbf{v}_s = B \, \mathbf{u}_f
\end{equation}

Se $B$ è costruito come media pesata con funzioni di forma $\phi_k$:
\begin{equation}
B_{ij} = \frac{\int_{\Omega_s} \phi_i(\mathbf{x}) \, \delta(\mathbf{x} - \mathbf{x}_j^f) \, dV}{\int_{\Omega_s} \phi_i(\mathbf{x}) \, dV}
\end{equation}
dove la normalizzazione per righe rende $\sum_j B_{ij} = 1$, allora $B$ è \textbf{adimensionale}.

\textbf{Problema:} Quando si usa $B^T$ per trasferire le forze:
\begin{equation}
\mathbf{F}_f = B^T \, \boldsymbol{\lambda}
\end{equation}
la forza trasferita $\mathbf{F}_f$ ha le stesse unità di $\boldsymbol{\lambda}$, cioè [N] (forza concentrata sulla cella fluido).

\subsubsection{Scalatura per il metodo di proiezione}

Nel metodo di proiezione esplicito, l'equazione della quantità di moto per la velocità intermedia è:
\begin{equation}
\mathbf{u}^* = \mathbf{u}^n + \Delta t \left[ -(\mathbf{u}^n \cdot \nabla)\mathbf{u}^n + \nu \nabla^2 \mathbf{u}^n + \frac{\mathbf{f}}{\rho_f} \right]
\end{equation}
dove $\mathbf{f}$ è una \textbf{forza per unità di volume} [N/m³].

\textbf{Passaggi per la scalatura corretta:}

\begin{enumerate}
    \item \textbf{Forza nodale dal solido:} $\boldsymbol{\lambda} = K_s \, \mathbf{d}_s$ \quad [N]
    
    \item \textbf{Trasferimento al fluido:} $\mathbf{F}_f = B^T \, \boldsymbol{\lambda}$ \quad [N] (forza sulla cella fluido)
    
    \item \textbf{Conversione in forza per unità di volume:}
    \begin{equation}
    \boxed{\mathbf{f} = \frac{\mathbf{F}_f}{V_{\text{cella}}} = \frac{\mathbf{F}_f}{\Delta x \cdot \Delta y} \quad \text{[N/m³] in 2D}}
    \end{equation}
    oppure $\mathbf{f} = \mathbf{F}_f / (\Delta x \cdot \Delta y \cdot \Delta z)$ in 3D.
    
    \item \textbf{Applicazione nella velocità intermedia:}
    \begin{equation}
    \mathbf{u}^* = \mathbf{u}^n + \Delta t \left[ \text{altri termini} + \frac{\mathbf{f}}{\rho_f} \right]
    \end{equation}
\end{enumerate}

\subsubsection{Formula finale per l'implementazione}

Combinando tutti i fattori, la correzione di velocità dovuta alla forza del solido è:
\begin{equation}
\boxed{\Delta \mathbf{u} = -\frac{\Delta t}{\rho_f \cdot \Delta x \cdot \Delta y} \, B^T \, (K_s \, \mathbf{d}_s)}
\end{equation}

Il segno negativo dipende dalla convenzione: se $\boldsymbol{\lambda}$ rappresenta la forza che il solido esercita \emph{sul} fluido, il segno è positivo; se rappresenta la reazione, è negativo.

\subsubsection{Caso alternativo: $B$ non normalizzato}

Se l'operatore $B$ è costruito \emph{senza} normalizzazione per righe, includendo i pesi di quadratura e il Jacobiano:
\begin{equation}
B_{ij} = \sum_q w_q \, \phi_i(\mathbf{x}_q) \, J_q \, \delta(\mathbf{x}_q - \mathbf{x}_j^f)
\end{equation}
allora $B_{ij}$ ha unità di [m²] (area) in 2D. In questo caso:
\begin{itemize}
    \item $\mathbf{v}_s = B \, \mathbf{u}_f$ richiederebbe normalizzazione per l'area nodale
    \item $\mathbf{F}_f = B^T \, \boldsymbol{\sigma}$ dove $\boldsymbol{\sigma}$ è uno \emph{stress} [N/m²] darebbe direttamente una forza [N]
\end{itemize}

È fondamentale essere consistenti tra la costruzione di $B$ e l'interpretazione delle quantità trasferite.

\subsection{Riferimenti}

Questa formulazione è consistente con i lavori su:
\begin{itemize}
    \item Immersed Finite Element Method (IFEM) - Zhang, Liu, Gay (2004)
    \item Distributed Lagrange Multiplier/Fictitious Domain (DLM/FD) - Glowinski, Pan, Hesla, Joseph (1999)
    \item Immersed Boundary Method con massa - Peskin, Printz (1993)
\end{itemize}

\end{document}
